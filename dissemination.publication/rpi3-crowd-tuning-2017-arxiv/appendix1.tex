%CK_APPENDIX={A}%
\section{Artifact Appendix}
\label{artifact_appendix}

\textbf{Submission guidelines:}\begin{center}{\it \href{http://ctuning.org/ae/submission-20161020.html}{cTuning.org/ae/submission-20161020.html}}\end{center}

This is an example of an Artifact Appendix which we introduced 
at the computer systems conferences including CGO, PPoPP, PACT and SuperComputing
to gradually unify artifact evaluation, sharing and reuse~\cite{ctuning-ae1,childers2016artifact,new_pub_model,Fur2009}.
%
We briefly describe how to install and use our autotuning workflow, visualize optimization results and reproduce them.
%
We also shared all scripts which we used to generate data and
graphs in all sections from this report though we did not yet 
have time to thoroughly document them.
%
In fact, we plan to gradually document them and standardize APIs 
of shared CK modules with the help of the community and motivated students.

%%%%%%%%%%%%%%%%%%%%%%%%%%%%%%%%%%%%%%%%%%%%%%%%%%%%%%%%%%%%%%%%%%%%%
\subsection{Abstract}

We provided the whole Collective Knowledge workflow with all dependencies
for collaborative, customizable, multi-dimensional and multi-objective autotuning 
of realistic workloads on Raspberry Pi 3 and other devices.

Current optimization results are available for GCC 7.1.0 (\href{http://cknowledge.org/repo/web.php?wcid=8289e0cf24346aa7:79bca2b76876b5c6}{link})
and for GCC 4.9.2 (\href{http://cknowledge.org/repo/web.php?wcid=8289e0cf24346aa7:d24a4fde9f120e10}{link}).
They are also available as a \href{https://github.com/ctuning/ck-rpi-optimization-results}{CK repository} 
and can be replayed on another platform via CK.

%%%%%%%%%%%%%%%%%%%%%%%%%%%%%%%%%%%%%%%%%%%%%%%%%%%%%%%%%%%%%%%%%%%%%
\subsection{Description}

\subsubsection{Check-list (artifact meta information)}

{\small
\begin{itemize}
  \item {\bf Algorithm:} -
  \item {\bf Program:} shared programs from the \href{https://github.com/ctuning/ctuning-programs}{CK ctuning-programs repository}
  \item {\bf Compilation:} any GCC
  \item {\bf Transformations:} compiler flag optimizations
  \item {\bf Binary:} will be produced during autotuning
  \item {\bf Data set:} real inputs from the \href{https://github.com/ctuning/ctuning-datasets-min}{CK ctuning-datasets-min repository}
  \item {\bf Run-time environment:} Raspbian (or any other)
  \item {\bf Hardware:} Raspberry Pi 3 (or any other)
  \item {\bf Run-time state:} will be monitored by CK (CPU frequency)
  \item {\bf Execution:} empirical measurements of the execution time of autotuned workloads via CK workflow
  \item {\bf Output:} best combinations of GCC compiler flags that improve execution time and code size 
  \item {\bf Experiment workflow:} autotuning, crowd-tuning  and collaborative machine learning workflow implemented using CK framework
  \item {\bf Experiment customization:} standard customization via CK API: select compiler, programs and data sets for autotuning, crowd-tuning and predictive modeling
  \item {\bf Publicly available?:} yes - CK autotuning and machine learning workflow (available under BSD 3-clause license) and all related artifacts are shared as reusable and customizable components via GitHub. 
\end{itemize}
}

\subsubsection{How software can be obtained (if available)}

You can obtain CK repositories with optimization results, shared programs and data sets, workflow for autotuning and crowd-tuning as following:

\begin{flushleft}
\texttt{\$ sudo pip install ck \newline
\$ ck pull repo:ck-rpi-optimization-results}
\end{flushleft}

Note that you may need around 1GB of free space. You can install 2 additional CK repositories~\cite{fursin_lokhmotov_upton_2018} 
from the public FigShare repository as following (need ~3GB of free space):

\begin{flushleft}
\texttt{\$ ck add repo:ck-rpi-optimization-results-reactions --zip=https://ndownloader.figshare.com/files/10218435 --quiet \newline
\$ ck add repo:ck-rpi-optimization-results-reactions2 --zip=https://ndownloader.figshare.com/files/10218441 --quiet \newline
\$ ck ls experiment:rpi3-* \newline}
\end{flushleft}

These repositories are so large because they contain all experiments from this report in a reproducible way
(we also plan to considerably reduce this size by removing duplicate information in the future). 
But if you want to prepare and run your own repositories you will likely need less than 100MB. 
See this artifact in the CK from the ACM CGO'17 paper~\cite{Ainsworth:2017:SPI:3049832.3049865} as example: 
\href{https://github.com/SamAinsworth/reproduce-cgo2017-paper}{github.com/SamAinsworth/reproduce-cgo2017-paper}

\subsubsection{Hardware dependencies}

Tested on Raspberry Pi Model B 3 devices with 4-core BCM2709 processor but should work on any platform.

\subsubsection{Software dependencies}

\begin{itemize}
  \item Raspbian GNU/Linux 8 (jessie)
  \item Collective Knowledge Framework~\cite{ck,ck-date16}
  \item Python 2.7+ or 3.4+
  \item Git client
  \item GCC 4.9.2 or GCC 7.1.0
\end{itemize}

\subsubsection{Data sets}

A minimal set of inputs for cTuning benchmarks available from the \href{https://github.com/ctuning/ctuning-datasets-min}{CK ctuning-datasets-min repository}.

%%%%%%%%%%%%%%%%%%%%%%%%%%%%%%%%%%%%%%%%%%%%%%%%%%%%%%%%%%%%%%%%%%%%%
\subsection{Installation}

Installation is performed using CK with the help of integrated cross-platform package manager~\cite{ck-env}:

\begin{flushleft}
\texttt{\$ sudo pip install ck \newline
\$ ck pull repo:ck-rpi-optimization-results \newline
\$ ck compile program:zlib --speed \newline}
\end{flushleft}

CK will automatically detect required software which is already installed on your platform, 
install missing packages, and prepare autotuning workflow for execution.

Note that CK allows multiple versions of different software to natively co-exist.
%
Therefore, you can install several versions of GCC which will be automatically
detected by CK and their environment prepared accordingly.
%
For example, you can install (build) GCC 7.1.0 on RPi 3 via CK as following:

\begin{flushleft}
\texttt{\$ ck pull repo:ck-dev-compilers \newline
\$ ck install package:compiler-gcc-any-src-linux-no-deps --env.PARALLEL\_BUILDS=1 --env.GCC\_COMPILE\_CFLAGS=-O0 --env.GCC\_COMPILE\_CXXFLAGS=-O0 --env.EXTRA\_CFG\_GCC=--disable-bootstrap --env.RPI3=YES --force\_version=7.1.0 \newline
\$ ck show env --tags=gcc}
\end{flushleft}

Note that you may need to install extra dependencies including

\begin{flushleft}
\texttt{\$ sudo apt-get install texinfo build-essential libgmp-dev libmpfr-dev libisl-dev libcloog-isl-dev libmpc-dev}
\end{flushleft}

You may also want to increase swap size on RPi 3 to speed up GCC building. 
You can change "CONF\_SWAPSIZE=100" in /etc/dphys-swapfile to "CONF\_SWAPSIZE=1000". 
But do not forget to change it back after successful build to avoid damaging your SD card.

%%%%%%%%%%%%%%%%%%%%%%%%%%%%%%%%%%%%%%%%%%%%%%%%%%%%%%%%%%%%%%%%%%%%%
\subsection{Experiment workflow}

%%%%%%%%%%%%%%%%%%%%%%%%%%%%%%%%%%%%%%%%%%%%%%%%%%%
\textbf{Autotuning example}

You can run zlib autotuning via CK as following:

\begin{flushleft}
\texttt{\$ ck autotune program:zlib --iterations=150 --repetitions=3 --scenario=9d88674c45b94971 --cmd\_key=decode --record\_uoa=my-first-experiment}
\end{flushleft}

CK will automatically detect available compilers, will ask user to select data set, 
and will evaluate 150 combinations of random compiler flags 
(repeating each experiment 3 times for statistical analysis 
of empirical variation of results).

Experimental results will be aggregated in a CK entry "experiment:my-first-experiment" a local CK repository:

\begin{flushleft}
\texttt{\$ ck find experiment:my-first-experiment}
\end{flushleft}

You can plot graph (execution time vs binary size) or view results in a web browser as following:

\begin{flushleft}
\texttt{\$ ck plot graph:my-first-experiment \newline
\$ ck browser experiment:my-first-experiment}
\end{flushleft}

You can compile and run zlib program via CK as following:
\begin{flushleft}
\texttt{\$ ck compile program:zlib --flags="some flags" \newline
\$ ck run program:zlib}
\end{flushleft}

Finally, you can participate in GCC crowd-tuning as following:
\begin{flushleft}
\texttt{\$ ck crowdsource program.optimization --gcc}
\end{flushleft}

%%%%%%%%%%%%%%%%%%%%%%%%%%%%%%%%%%%%%%%%%%%%%%%%%%%%%%%%%%%%%%%%%%%%%
\subsection{Evaluation and expected result}

You can find all scripts to perform experiments from this article as following:

\begin{flushleft}
\texttt{\$ ck ls ck-rpi-optimization-results:script:* | sort}
\end{flushleft}

You can then go to each individual entry and see related scripts:
\begin{flushleft}
\texttt{\$ ls `ck find script:rpi3-susan-autotune`}
\end{flushleft}

You can find all experimental results in the following entries:
\begin{flushleft}
\texttt{\$ ck ls ck-rpi-optimization-results:experiment:* | sort}
\end{flushleft}

You can then browse all results in your web browser as following:
\begin{flushleft}
\texttt{\$ ck browser experiment:rpi3-zlib-decode-gcc4-150b-rnd-frontier}
\end{flushleft}

You can find information about how to replay each autotuning iteration there, for example:
\begin{flushleft}
\texttt{\$ ck replay experiment:b0f31c56475aa510 --point=46049203405c5347}
\end{flushleft}

CK should normally show expected and new results while reporting any unexpected 
behavior (if difference is more than some threshold such as 5\%).

%%%%%%%%%%%%%%%%%%%%%%%%%%%%%%%%%%%%%%%%%%%%%%%%%%%%%%%%%%%%%%%%%%%%%
\subsection{Experimental methodology}
\label{experimental_methodology}

One of the most important points of using Collective Knowledge framework
is to take advantage of the experimental methodology for computer systems research
continuously improved by the community.
%
For this purpose, we instrument programs using small xOpenME library
which allows us to monitor behavior of some code regions and dump 
final statistics to a JSON file in the CK format.
%
CK will then repeat each autotuning iteration N times,
apply statistical analysis on all exposed characteristics, report min, max and mean values,
and calculate expected value based on a histogram of all results (if supported by used Python)
as shown in Figure~\ref{fig:reproducibility} (a).

We then calculate improvements of a given optimization over reference one (-O3)
using minimal and expected execution times, and record differences.
%
If the difference is more than 5\%, we mark such experient is noise and untrustable 
to be analyzed and improved later by the community.
%
If several system states are detected as shown in Figure~\ref{fig:reproducibility} (b),
CK will not be able to reproduce them - it then means that the common CK experimental workflow
should also be improved for this hardware and environment to be able to distinguish such
states (such as CPU and GPU frequency due to DVFS for example).

%%%%%%%%%%%%%%%%%%%%%%%%%%%%%%%%%%%%%%%%%%%%%%%%%%%%%%%%%%%%%%%%%%%%%
\subsection{Notes}
\label{notes}

We did not have time to thoroughly document experiments from sections~\ref{sec:crowdfuzzing}+ of this report.
However we shared all CK modules, workflows and scripts we used in this report 
in the following CK entries:

\begin{flushleft}
\texttt{\$ ck ls script:rpi3-*} \newline
\texttt{\$ ck ls converting-ad-hoc-works-to-ck-*} \newline
\end{flushleft}

Scripts from Sections~\ref{sec:autotuning} and \ref{sec:flag_autotuning}
to invoke portable and customizable CK autotuning workflow:

\begin{flushleft}
\texttt{\$ ck find script:rpi3-susan-autotune} \newline
\texttt{\$ ck find script:rpi3-susan-graphs} \newline
\texttt{\$ ck find script:rpi3-susan-reduce} \newline
\texttt{\$ ck find script:rpi3-all-autotune} \newline
\end{flushleft}

Scripts from Section~\ref{sec:crowdtuning}:

\begin{flushleft}
\texttt{\$ ck find script:rpi3-all-autotune} \newline
\texttt{\$ ck find script:rpi3-crowdtune} \newline
\end{flushleft}

Scripts from Section~\ref{sec:collaborative}:

\begin{flushleft}
\texttt{\$ ck find script:rpi3-zlib-decode-autotune} \newline
\texttt{\$ ck find script:rpi3-zlib-decode-graphs} \newline
\texttt{\$ ck find script:rpi3-zlib-decode-reduce} \newline
\texttt{\$ ck find script:rpi3-zlib-encode-autotune} \newline
\texttt{\$ ck find script:rpi3-zlib-encode-graphs} \newline
\texttt{\$ ck find script:rpi3-zlib-encode-reduce} \newline
\end{flushleft}

Scripts from Section~\ref{sec:crowdfuzzing}:

\begin{flushleft}
\texttt{\$ ck find script:rpi3-susan-fuzz-bugs} \newline
\end{flushleft}

Scripts from Sections~\ref{sec:crowdmodeling} and \ref{sec:features}:

\begin{flushleft}
\texttt{\$ ck find script:rpi3-crowdmodel} \newline
\end{flushleft}

Scripts from Section~\ref{sec:datasets}: //input-aware

\begin{flushleft}
\texttt{\$ ck find script:rpi3-all-autotune-multiple-datasets} \newline
\texttt{\$ ck find script:rpi3-input-aware-autotune-blas} \newline
\end{flushleft}

Scripts from Section~\ref{sec:competitions}:

\begin{flushleft}
\texttt{\$ ck find script:converting-ad-hoc-works-to-ck-slambench-autotuning} \newline
\end{flushleft}


%%%%%%%%%%%%%%%%%%%%%%%%%%%%%%%%%%%%%%%%%%%%%%%%%%%%%%%%%%%%%%%%%%%%%
\subsection{Conclusion}

We hope that our customizable autotuning and machine learning workflow 
can teach students, scientists and engineers learn how to collaboratively
co-design Pareto-efficient software and hardware stack for emerging workloads.
%
Please feel free to send us updates and patches to fix, help us to improve or extend
our artifacts with documentation, and keep in touch with our community via
CK mailing list:~\href{https://groups.google.com/d/forum/collective-knowledge}{groups.google.com/d/forum/collective-knowledge}!
