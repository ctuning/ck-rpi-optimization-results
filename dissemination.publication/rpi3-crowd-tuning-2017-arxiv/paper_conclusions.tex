Researchers are now in a race to bring artificial intelligence to all possible
devices from IoT to supercomputers which will require 
much more efficient software and hardware then currently available.
%
At the same time, computer engineers have already been struggling for many years 
to develop efficient sub-components of computer systems including
algorithms, compilers and run-time systems.

The major issues including raising complexity, lack of a common experimental framework 
and lack of practical knowledge exchange between academia and industry.
%
Rather than innovating, researchers have to spend more and more time 
writing their own, ad-hoc and not easily customizable support tools 
to perform experiments such as multi-objective autotuning.


We presented our long-term educational initiative to teach
students and researchers how to solve the above problems 
using customizable workflow frameworks similar to other sciences.
%
We showed how to convert ad-hoc, multi-objective
and multi-dimensional autotuning into a portable and customizable workflow 
based on open-source Collective Knowledge workflow framework.
%
We then demonstrated how to use it to implement various scenarios
such as compiler flag autotuning of benchmarks and realistic workloads
across Raspberry Pi 3 devices in terms of speed and size.
%
We also demonstrated how to crowdsource such autotuning across different
devices provided by volunteers similar to SETI@home, collect the most efficient optimizations
in a reproducible way in a public repository of knowledge at ~\href{http://cknowledge.org/repo}{cKnowledge.org/repo}, 
apply various machine learning techniques including decision trees, the nearest neighbor classifier
and deep learning to predict the most efficient optimizations for previously
unseen workloads, and then continue improving models and features
as a community effort.
%
We now plan to develop an open web platform together with the community
to provide a user-friendly front-end to all presented workflows 
while hiding all complexity.

We use our methodology and open-source CK workflow framework and repository
to teach students how to exchange their research artifacts and results 
as reusable components with a a unified API and meta-information,
perform collaborative experiments, automate Artifact Evaluation
at journals and conferences~\cite{ctuning-ae1}, build upon each others' work,
make their research more reproducible and sustainable, 
and eventually accelerate transfer of their ideas to industry.
%
Students and researchers can later use such skills and unified artifacts
to participate in our open ReQuEST tournaments on reproducible and Pareto-efficient
co-design of the whole software and hardware stack for emerging workloads
such as deep learning and quantum computing in terms of
speed, accuracy, energy and costs~\cite{request}.
%
This, in turn, should help the community build an open repository of 
portable, reusable and customizable algorithms continuously optimized
across diverse platforms, models and data sets
to assemble efficient computer systems
and accelerate innovation.
